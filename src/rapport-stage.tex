\documentclass[11pt,a4paper]{rapport-stage-umons}

\usepackage[utf8]{inputenc}
\usepackage[T1]{fontenc}
\usepackage[francais]{babel}
\usepackage{amssymb,amsmath,amsthm}
\usepackage{graphicx}
\usepackage{hyperref}

%\usepackage{hyperref}% hyperliens dans le PDF, pas pour impression

\title{Stage chez Be Sport / Ocsigen}
\author{Danny \textsc{Willems}}
\date{2016--2017}
\directeur{Christophe \textsc{Troestler}}
\maitre{Vincent \textsc{Balat}}
\discipline{Mathématiques}
\institut{Département de mathématiques}

%%%%%%%%%%%%%%%%%%%%%%%%%%%%%%%%%%%%%%%%%%%%%%%%%%%%%%%%%%%%%%%%%%%%%%%%
%% Vos macros


%%%%%%%%%%%%%%%%%%%%%%%%%%%%%%%%%%%%%%%%%%%%%%%%%%%%%%%%%%%%%%%%%%%%%%%%

% Compile uniquement certains morceaux sans perdre les références
% automatiques et la table des matières des parties déjà compilées :
%\includeonly{introduction,chapitre1}

\begin{document}
% Éventuellement utiliser l'environnement « preface » pour avoir une
% numérotation des pages en chiffres romains.

\tableofcontents

\section{Présentation de l'entreprise}

\subsection{Be Sport}

Be Sport\cite{besport} est une entreprise française développant un réseau social
et un média centrés sur le sport. Initialement lancé en 2012 à San Francisco
aux États-Unis, Be Sport s'est implanté à Paris fin 2015 sous la
direction de Philippe Robert, principal investisseur, et de membres de l'équipe d'Ocsigen.

Contrairement à la grande majorité des plateformes du sport qui se \\ concentrent
sur les professionnels, Be Sport est une plateforme regroupant tous les
types d'acteurs du sport: amateurs, professionnels, clubs, fédérations, médias,
etc.

Avec Be Sport, il est possible de

\begin{itemize}
  \item suivre des sportifs, amateurs comme professionnels, des clubs et des
    fédérations ;
  \item se connecter avec d'autres personnes inscrites sur la plateforme ;
  \item créer des événements sportifs (par exemple des tournois, des concours,
    des championnats) et inviter ses connexions ;
  \item discuter grâce à un tchat ;
  \item suivre les dernières nouvelles des sports qui nous intéressent ;
  \item partager des nouvelles, des résultats ;
  \item et bien plus : de nouvelles fonctionnalités sont ajoutées très régulièrement.
\end{itemize}

Le trait de Be Sport qui m'a attiré pour le choix de mon stage est la
technologie utilisée: OCaml et en particulier le framework web Ocsigen.

\subsection{Ocsigen}

Ocsigen\cite{ocsigen-website} est le plus connu des frameworks web écrit en
OCaml.

Lancé en fin d'année 2004 par Vincent Balat, Ocsigen\footnote{Pour \og OCaml site
  generator \fg.} souhaite apporter des solutions, innovantes à ses débuts, à
plusieurs problématiques du développement d'applications web:

\begin{enumerate}
  \item utiliser un seul et unique langage (OCaml) pour développer la partie cliente et
    la partie serveur. En effet, la plupart des architectures utilisent deux
    langages différents. Cela implique qu'un développeur web doit apprendre différents
    langages. Du coté de l'entreprise, cela peut aussi impliquer de devoir
    embaucher deux développeurs: un pour le coté client, souvent appelé
    \og développeur front-end \fg, et un autre pour le coté serveur, souvent appelé
    \og développeur back-end \fg.
  \item décrire, dans un seul fichier, les morceaux de code qui seront
exécutés coté client et ceux qui seront exécutés coté serveur. Ceci est réalisé
à travers une extension de syntaxe PPX d'OCaml\footnote{Gabriel Radanne, aka
  Drup, travaille sur un compilateur Eliom, voir
  \cite{ocsigen-eliomlang-github}. Ce compilateur fournit plus de possibilités
  que PPX.}.
  \item injecter du code serveur dans la partie cliente.
    L'utilité de cette fonctionnalité est de laisser le serveur évaluer
des expressions pour décharger le client de certains calculs qui pourraient être
lourds.
  \item typer statiquement (au moment de la compilation, non à l'exécution), les
    programmes (coté serveur et coté client) ainsi que le code injecté.
  \item utiliser les bibliothèques du langage de base (OCaml), que cela
    soit coté client ou coté serveur.
\end{enumerate}

Bien que la plupart des points soient résolus indépendemment par plusieurs
technologies (Node.js\cite{nodejs-website}, Meteor\cite{meteor-website},
Scala\cite{scala-website} + Play\cite{play-website} +
ScalaJS\cite{scalajs-website}, entre autres), aucune technologie unique et
connue du grand public, répond à tous les besoins.

Le projet Ocsigen répond à chacun de ces besoins, et même plus. Le projet est
composé de plusieurs sous projets dont

\begin{itemize}
  \item Ocsigen Server\cite{ocsigen-server-github}, un serveur web écrit entièrement en OCaml.
  \item un compilateur de bytecode OCaml vers JavaScript, js\_of\_ocaml\cite{ocsigen-js-of-ocaml-github}, qui
    pemet d'écrire du code OCaml et de l'exécuter dans le navigateur web. En
plus du compilateur, js\_of\_ocaml fournit un binding OCaml assez complet (bien
que bas-niveau) au DOM et à la librairie standard JavaScript. Une particularité
de js\_of\_ocaml comparé à d'autres compilateurs/transpileurs OCaml vers
JavaScript est le support de tout le langage OCaml, sans syntaxe particulière.
  \item une bibliothèque implémentant le standard XML de manière typée en OCaml,
TyXML\cite{ocsigen-tyxml-github}. Cela permet d'écrire en particulier du HTML
typé et de respecter les normes du W3C.
  \item Eliom, qui permet d'écrire, avec un haut niveau
    d'abstraction et des concepts innovants, des applications web
client-serveur. Eliom dépend de Ocsigen Server, TyXML et js\_of\_ocaml et
fournit une abstraction plus haut niveau des API de ces différents projets.
  \item Ocsigen Toolkit, un ensemble de widgets web (différents boutons, des
    menus, des icones de chargements, etc) permettant d'utiliser les
    particularités d'exécution client-serveur d'Eliom.
  \item Ocsigen Start qui fournit un template Eliom avec un ensemble de
    bibliothèques pour gérer des utilisateurs, des groupes d'utilisateurs, des
    notifications push, etc. Ocsigen Start permet également de développer des
    applications mobiles, le tout en OCaml.
\end{itemize}

Ayant passé beaucoup de temps à apprendre et à travailler sur Eliom et Ocsigen
Start, l'apprentissage et la maitrise de ces deux bibliothèques ont eu une
grande importance lors de mon stage. 

\subsubsection{Eliom}

Le but principal d'Ocsigen est d'unifier le développement de la partie serveur
et la partie cliente en utilisant un langage unique et typé statiquement.

Eliom est le centre du projet Ocsigen (avec comme dépendances Ocsigen Server,
TyXML et js\_of\_ocaml) et constitue la solution aux besoins donnés ci-dessus.
Eliom fournit également une interface de haut
niveau pour des besoins usuels en programmation web comme un système de
notifications \\ (\verb|Eliom_notif|), de création d'HTML réactif
(\verb|Eliom_shared.React.S|) et des données réactives
(\verb|Eliom_shared.ReactiveData|), d'échange de données entre clients
(\verb|Eliom_bus|), de créations de services pour les routes de l'application et
des enregistrements de ces services (\verb|Eliom_service| et \\
\verb|Eliom_registration|), un système de gestion de cookies
\\ (\verb|Eliom_cookies|), création de formulaires typés
(\verb|Eliom_form|), un système de RPC (en utilisant la fonction
\verb|Eliom_client.server_function|) et bien plus.

De plus, depuis Eliom 6, il est possible d'écrire des applications mobiles.

En utilisant PPX, Eliom ajoute de la syntaxe au langage OCaml usuel.\footnote{Pour plus
d'informations sur le langage Eliom, voir \cite{gabriel-radanne-paper-eliom}.}
Eliom fournit également des outils pour simplifier la compilation:
\begin{itemize}
  \item \verb|eliomc| pour compiler du code serveur en bytecode.
  \item \verb|eliomopt| pour compiler du code serveur en natif.
  \item \verb|js_of_eliom| pour compiler en JavaScript le code client.
\end{itemize}

Pour différentier les fichiers contenant du code Eliom et ceux qui n'en \\
contiennent pas, de nouvelles extensions de fichiers sont utilisées: \verb|eliomi|
pour l'équivalent de \verb|mli| et \verb|eliom| pour \verb|ml|.

Pour définir une expression qui sera évaluée coté serveur (resp. coté client), la syntaxe
\verb|let%server v = 42| (resp. \verb|let%client v = 42|) est utilisée. Il
existe aussi la syntaxe \verb|let%shared v = 42| pour définir en même temps une
même expression coté client et coté serveur. L'équipe d'Ocsigen travaille
également sur un nouveau compilateur Eliom, voir
\cite{ocsigen-eliomlang-github}, qui fournit, entre autres, la même extension
pour les types (\verb|type%server t = int|) et le langage de modules
(\verb|module%server M = struct type t = int end|), non disponibles actuellement avec PPX.

Il est également possible de définir des valeurs coté serveur et d'y accéder
coté client avec la syntaxe \verb|let%client v = ~%server_value| où \\
\verb|server_value| est une valeur définie coté serveur.

Une dernière possibilité avec Eliom est, à l'inverse d'évaluer une expression
coté serveur et la récupérer coté client, il est possible, dans une expression
coté serveur, de calculer une expression coté client et d'en récupérer le
résultat coté serveur.
La syntaxe utilisée est \verb|let%server x = [%client 5 + 2]|.

Si plusieurs expressions doivent être évaluées coté serveur (resp. coté client),
il est possible d'utiliser la syntaxe \verb|[%%server expressions]| (resp. \\
\verb|[%%client expressions]|) dans les fichiers \verb|eliom| et la syntaxe \\
\verb|[%%server.start] expressions| (resp. \\ \verb|[%%client.start] expressions|)
dans les fichiers \verb|eliomi|.

\subsubsection{Ocsigen Start}

Anciennement eliom-base-app\footnote{Le nom fut changé pendant la période de
  mon stage.}, le projet Ocsigen Start\cite{ocsigen-start-github} est la base de
l'application Be Sport, distribué sur GitHub sous licence LGPL.

A travers ce projet, l'équipe d'Ocsigen souhaite fournir un ensemble de
fonctions permettant de créer très facilement un prototype d'une application web
et mobile. Cela implique, en partie, de gérer très facilement une base de
données d'utilisateurs, de groupes, d'emails, fournir des services usuels
(inscriptions d'utilisateurs, envoi d'email de confirmation, connexion,
modification de données personnelles), permettre de gérer plusieurs emails par
utilisateurs, le redimensionnement et cropping de photos et fournir des systèmes
de sessions et de cache.

En plus de la bibliothèque, Ocsigen Start fournit un template qui montre
comment utiliser les diverses fonctionnalités d'Eliom comme les
RPC\footnote{Remote Procedure Calls}, les notifications à travers
\verb|Eliom_notif| mais aussi les fonctions de la bibliothèque.

Le template fourni permet également d'avoir une application mobile (Android,
iOS, Windows Phone) de l'application web développée. Le framework web et mobile
Cordova est utilisé pour l'application Android et iOS tandis que le concept
d'Universal Web App est utilisé pour Windows Phone\footnote{Ceci en raison de
  la faible compatibilité des plugins Cordova sous Windows Phone.}.

Ocsigen Start réalise des choix pour les développeurs comme PostgreSQL pour la
base de données ou SASS (pré-processeur CSS) pour le style.
Dans les prochaines versions, Ocsigen Start devrait être compatible avec
différents gestionnaires de bases de données comme MySQL.
\section{Tâches effectuées}

Durant mon stage, j'ai réalisé différentes tâches touchant à différents
domaines. Ces tâches, j'ai eu la liberté de les choisir (en fonction des besoins
de Be Sport et de la répartition de celles-ci) au fur et à mesure des quatres
mois passés chez Be Sport. Certaines tâches, que je qualifierai de mineures et
majoritairement dans Ocsigen Start, étaient ajoutées aux tâches principales.

Pendant tout mon stage, je n'ai jamais contribué directement au code de la
plateforme Be Sport: le travail que j'effectuais était principalement dans Ocsigen Start
(que Be Sport utilise comme base) ou dans un projet séparé (comme pour les push
notifications) et celui-ci était utilisé par un développeur de Be Sport comme
base. Cette méthode a été choisie par Vincent Balat pour que je puisse être
productif plus rapidement car sinon, j'aurais du, en plus du projet Ocsigen qui
est déjà assez conséquent, comprendre l'ensemble de Be Sport.

\subsection{OAuth2.0 et OpenID Connect}

Lorsque je suis arrivé le premier jour, j'ai pu choisir une première tâche dans
une liste de prochaines fonctionnalités devant être développées dans Be Sport.

J'ai choisi l'implémentation d'un système semblable à Facebook
Login\cite{facebook-login} dans Be Sport, appelée Be Sport Connect\footnote{La dénomination n'a pas été officiellement
  choisie mais celle-ci était utilisée pour parler de cette fonctionnalité.}.
Cette fonctionnalité doit permettre à des applications indépendantes de Be Sport
de pouvoir utiliser un compte créé sur Be Sport sans que l'utilisateur donne, à
nouveau, certaines informations comme son adresse email, son nom, son prénom et
un mot de passe et sans pour autant accéder à
toutes les données possédées par Be Sport sur l'utilisateur (comme certains
sites font avec Facebook, LinkedIn, Twitter ou encore GitHub).\footnote{Par la
suite (après 6 semaines de stage), les
  priorités de Be Sport ont changé
  et j'ai donc laissé de côté cette partie pour me focaliser sur Ocsigen Start
  et les push notifications, tâches décrites ci-dessous. Vers la fin de mon
  stage, le travail a été repris pour mettre à jour avec les changements
  d'Ocsigen Start. Cependant, l'implémentation, bien que très avancée, n'est pas
encore finie.}.
L'idée derrière cette fonctionnalité est de séparer l'application Be Sport en
plusieurs applications indépendantes. Par exemple, le tchat Be Sport, actuellement
dépendant de Be Sport, pourrait être une application entièrement séparée de Be
Sport sous le nom de \emph{Be Sport Chat}. Un autre exemple est une application
\emph{Be Sport Location} qui permet de localiser un utilisateur connecté et
lister les différents événements proches de lui. Ces deux applications ont
besoin de créer des comptes pour retenir diverses informations, d'où l'intérêt
de Be Sport Connect pour éviter de devoir créer des comptes séparés pour les
deux applications en plus de Be Sport.


\subsection{JWT : Json Web Token}

Comme dit précédemment, OpenID Connect utilise le standard JWT (Json Web
Token)\cite{official-jwt-website, official-openid-connect-website, rfc-jwt}
pour représenter de manière sécurisée des demandes (\textbf{claims} en anglais) et les
transférer entre deux applications en utilisant des \textit{jetons} (ou
\textit{token} en anglais).
Aucune bibliothèque n'existant en OCaml pour gérer ce standard, j'ai du
l'implémenter. La bibliothèque est disponible sur le GitHub de Be
Sport\cite{ocaml-jwt-github} ainsi que
dans OPAM sous le nom \emph{jwt} et est distribué sous licence LGPL.

Un JSON Web Token est composé de trois éléments séparés par des points:
\begin{itemize}
  \item un \textit{header} comprenant le type du jeton (dans notre cas
    \emph{JWT}) et l'algorithme de chiffrement (par exemple
    \emph{HMAC-SHA-256}, identifié par \emph{HS256})
    représenté par un JSON. Le premier élément du JSON Web Token est le
    résultat de ce JSON en utilisant l'encodage base64\cite{rfc-base64}.
  \item les \textit{claims} ou le \textit{payload} comprenant les demandes ou
    des méta données devant être transférées. Le résultat est aussi représenté
    par un JSON encodé en utilisant l'encodage base64.
  \item la \textit{signature} qui rend ce standard sécurisé pour la transmission
    des données. Elle consiste en le résultat du chiffrement (par l'algorithme de
    chiffrement donné dans le header) de la concaténation
    des deux encodages base64 précédant séparé par un point. Si l'algorithme de
    chiffrement utilisé est \emph{HS256}, une clef secrète est utilisée pour
    chiffrer. La chaîne de caractères résultante du chiffrage est encodé en
    utilisant l'encodage base64 et représentera le dernier élément du jeton.
\end{itemize}

\begin{figure}
  \centering
  \includegraphics[width=300px]{jwt-example.png}
  \caption{Schéma expliquant la structure d'un JSON Web Token. Le jeton qui sera
    échangé entre les deux applications est la chaîne de caractère. Les parties header
    et claims résultent de l'encodage base64 des JSON respectifs. La signature
    est générée grâce à l'algorithme HS256 en utilisant une clef secrète qui
    n'est pas donnée dans ce cas.}
\end{figure}

Un exemple complet est donné sur le site officiel du standard.\cite{official-jwt-website}

La bibliothèque OCaml \emph{jwt}\cite{ocaml-jwt-github} fournit une interface
simple pour créer, chiffrer et déchiffrer un JWT.

Un token est représenté par un type abstrait \verb|t|. Le token chiffré peut
être généré grâce à la fonction \verb|token_of_t|. Le header et les différents
claims peuvent être reconstitués en JSON en utilisant la fonction
\verb|t_of_token|.

Au niveau des algorithmes de chiffrement,
\textit{Cryptokit}\cite{ocaml-cryptokit-ocaml-forge} est utilisé et uniquement
HS256 est supporté\footnote{Car uniquement cette méthode est utilisée pour OpenID
  Connect}. Les différents algorithmes sont contenus dans un type somme \verb|algorithm|.

Un header est représenté par un type abstrait \verb|header| et une valeur de ce
type peut être créé grâce à la fonction \verb|header_of_algorithm_and_type|.

Une claim est représentée par un type abstrait \verb|claim| et une liste de
claim usuel sont déjà définis (comme \emph{iss}, \emph{sub} et \emph{exp} qui
sont utilisés dans OpenID Connect). De nouveau claims peuvent être créées en
utilisant la fonction \verb|claim| qui prend le nom en paramètre. 

Le payload, c'est-à-dire l'ensemble des claims, peut être créé à travers la
valeur \verb|empty_payload| et la fonction \verb|add_claim|.
Pour obtenir les différentes claims d'un payload, la fonction \verb|find_claim|
peut être utilisée.

\subsection{Crawling de données et binding à NodeJS et NightmareJS}

\subsection{Ocsigen Start}


\subsubsection*{Push notifications}

\subsubsection*{i18n}

\subsubsection*{Tâches mineures}

\subsection{Personnelles}

A côté des heures passées dans les bureaux de Be Sport, j'ai continué à
m'intéresser au projet Ocsigen et à y contribuer. Bien que ces tâches soient
rapides, elles m'ont permis d'améliorer mes connaissances et d'être plus efficace
les jours suivants. Voici une liste non exhaustive:
\begin{itemize}
  \item eliom-distillery : option \verb|-list-templates| afin de lister tous les templates installés.
  \item eliom-distillery : option \verb|-y| pour ne pas demander de confirmation lors
    de la création d'un projet.
  \item eliom-distillery : options \verb|-add-git-template| et
\verb|-rm-template| pour permettre d'ajouter ses propres templates Eliom.\cite{eliom-distillery-repo}
  \item Binding OCaml au plugin Cordova
    \href{https://github.com/fechanique/cordova-plugin-fcm}{cordova-plugin-fcm}.\cite{ocaml-cordova-plugin-fcm} \\
    Celui-ci s'ajoute à la liste des bindings que j'ai réalisés pendant
l'année 2015-2016.\cite{ocaml-cordova-plugin-list}
\end{itemize}

\section{Apport du projet Erasmus+}

Parler de la HackerHouse, du fait que j'étais dans un environnement où on
parlait startup chaque jour, où on rencontrait des entrepreneurs chaque
week-end, où on organisait des hackathons assez régulièrement.

Paris est également un endroit pour entreprendre et être au centre de
l'évolution techonologique.
\section{Conclusion}

J'ai également pu remarquer que les librairies ou technologies maintenues par de grandes
sociétés ne sont pas nécessairement meilleures ou les plus à jour. (exemple:
PhoneGap et phonegap-plugin-push ou Google et la documentation de FCM et l'API
de Google Maps).

Arrêter de considérer le tout comme une boite noire.

\nocite{*}

% Si vous utilisez (conseillé) BibTeX pour votre bibliographie :
\bibliographystyle{acm}
\bibliography{rapport-stage} % si le fichier BibTeX est rapport-stage.bib

\end{document}
%%% Local Variables: 
%%% mode: latex
%%% TeX-master: t
%%% TeX-PDF-mode: t
%%% End: 
