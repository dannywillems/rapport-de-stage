\section{Présentation de l'entreprise}

\subsection{Be Sport}

Be Sport\cite{besport} est une entreprise française développant un réseau social
et un média centrés sur le sport. Initialement lancé en 2012 dans la Silicon
Valley aux États-Unis, Be Sport s'est implanté à Paris fin 2015 sous la
direction de Philippe Robert, principal investisseur, et de membres de l'équipe d'Ocsigen.

Contrairement à la grande majorité des plateformes du sport qui se concentrent
sur les professionnels, Be Sport est une plateforme regroupant tous les
types d'acteurs du sport: amateurs, professionnels, clubs, fédérations, médias,
etc.

Avec Be Sport, il est possible de

\begin{itemize}
  \item suivre des sportifs, amateurs comme professionnels, des clubs et des
    fédérations ;
  \item se connecter avec d'autres personnes inscrites sur la plateforme ;
  \item créer des événements sportifs (par exemple des tournois, des concours,
    des championnats) et inviter ses connexions ;
  \item discuter grâce à un tchat ;
  \item suivre les dernières nouvelles des sports qui nous intéressent ;
  \item partager des nouvelles, des résultats, etc.
\end{itemize}

Le trait de Be Sport qui m'a attiré pour le choix de mon stage est la
technologie utilisée: OCaml et en particulier le framework web Ocsigen.

\subsection{Ocsigen}

Ocsigen\cite{ocsigen-website} est le plus connu des frameworks web écrit en
OCaml.

Lancé en fin d'année 2004 par Vincent Balat, Ocsigen\footnote{Pour OCaml site
  generator.} souhaite apporter des solutions, innovantes à ses débuts, à
plusieurs problématiques du développement d'applications web:

\begin{enumerate}
  \item utiliser un seul et unique langage (OCaml) pour développer la partie cliente et
    la partie serveur. En effet, la plupart des architectures utilise deux
    langages différents. Cela implique qu'un développeur web doit apprendre différents
    langages. Du coté de l'entreprise, cela peut aussi impliquer de devoir
    embaucher deux développeurs: un pour le coté client, souvent appelé
    \og développeur front-end \fg, et un autre pour le coté serveur, souvent appelé
    \og développeur back-end \fg.
  \item pouvoir, dans un seul fichier, décrire les morceaux de code qui seront
exécutés coté client et ceux qui seront exécutés coté serveur. Ceci est réalisé
à travers une extension de syntaxe PPX d'OCaml.
  \item permettre le partage de code ainsi que la communication entre la partie
    cliente et la partie serveur.
  \item typer statiquement (au moment de la compilation, non à l'exécution), les
    programmes (coté serveur et coté client) ainsi que le code partagé entre les deux.
  \item pouvoir utiliser les bibliothèques du langage de base (OCaml), que cela
    soit coté client ou coté serveur.
\end{enumerate}

Bien que la plupart des points soit résolu indépendemment par plusieurs
technologies (Node.js\cite{nodejs-website}, Meteor\cite{meteor-website},
Scala\cite{scala-website} + Play\cite{play-website} +
ScalaJS\cite{scalajs-website}, entre autres), aucune technologie unique répond à tous les besoins.

Le projet Ocsigen répond à chacun de ces besoins, et même plus. Le projet est
composé de plusieurs sous projets dont

\begin{itemize}
  \item un serveur web écrit entièrement en OCaml (Ocsigen Server\cite{ocsigen-server-github}).
  \item un compilateur de bytecode OCaml vers JavaScript, js\_of\_ocaml\cite{ocsigen-js-of-ocaml-github}, qui
    pemet d'écrire du code OCaml et de l'exécuter dans le navigateur web. En
plus du compilateur, js\_of\_ocaml fournit un binding OCaml assez complet (bien
que bas-niveau) au DOM et à la librairie standard JavaScript. La particularité
de js\_of\_ocaml comparé à d'autres compilateurs/transpileurs OCaml vers
JavaScript est le support de tout le langage OCaml, sans syntaxe particulière.
  \item une bibliothèque implémentant le standard XML de manière typée en OCaml,
TyXML\cite{ocsigen-tyxml-github}. Cela permet d'écrire en particulier de l'HTML
typé et de respecter ainsi les normes W3C.
  \item Eliom, qui permet d'écrire, avec un haut niveau
    d'abstraction et des concepts innovants, des applications web
client-serveurs. Eliom dépend de Ocsigen Server, TyXML et js\_of\_ocaml et
fournit une abstraction plus haut niveau des API de ces différents projets.
  \item Ocsigen Toolkit, un ensemble de widgets web (différents boutons, des
    menus, des icones de chargements, etc) permettant d'utiliser les
    particularités d'exécution client-serveur d'Eliom.
  \item Ocsigen Start qui fournit un template Eliom avec un ensemble de
    bibliothèques pour gérer des utilisateurs, des groupes d'utilisateurs, des
    notifications push, etc. Ocsigen Start permet également de développer des
    applications mobiles, le tout en OCaml.
\end{itemize}

Ayant passé beaucoup de temps à apprendre et à travailler sur Eliom et Ocsigen
Start, l'apprentissage et la maitrise de ces deux bibliothèques ont eu une
grande importance lors de mon stage. 

\subsubsection*{Eliom}

Le but principal d'Ocsigen est d'unifier le développement de la partie serveur
et la partie cliente en utilisant un langage unique et typé statiquement.

Eliom est le centre du projet Ocsigen (avec comme dépendance Ocsigen Server et
js\_of\_ocaml) et constitue la solution aux besoins donnés ci-dessus.

\paragraph*{Séparation du code client et du code serveur}


\subsubsection*{Ocsigen Start}

Anciennement eliom-base-app\footnote{Le nom fut changé pendant la période de
  mon stage.}, le projet Ocsigen Start\cite{ocsigen-start-github} est la base de
l'application Be Sport, distribué sur GitHub sous licence LGPL.