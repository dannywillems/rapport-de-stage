\section{Présentation de l'entreprise}

\subsection{Be Sport}

Be Sport\cite{besport} est une entreprise française développant un réseau social
et un média centrés sur le sport. Initialement lancé en 2012 à San Francisco
aux États-Unis, Be Sport s'est implanté à Paris fin 2015 sous la
direction de Philippe Robert, principal investisseur, et de membres de l'équipe d'Ocsigen.

Contrairement à la grande majorité des plateformes du sport qui se \\ concentrent
sur les professionnels, Be Sport est une plateforme regroupant tous les
types d'acteurs du sport: amateurs, professionnels, clubs, fédérations, médias,
etc.

Avec Be Sport, il est possible de

\begin{itemize}
  \item suivre des sportifs, amateurs comme professionnels, des clubs et des
    fédérations ;
  \item se connecter avec d'autres personnes inscrites sur la plateforme ;
  \item créer des événements sportifs (par exemple des tournois, des concours,
    des championnats) et inviter ses connexions ;
  \item discuter grâce à un tchat ;
  \item suivre les dernières nouvelles des sports qui nous intéressent ;
  \item partager des nouvelles, des résultats ;
  \item et bien plus : de nouvelles fonctionnalités sont ajoutées très régulièrement.
\end{itemize}

Le trait de Be Sport qui m'a attiré pour le choix de mon stage est la
technologie utilisée: OCaml et en particulier le framework web Ocsigen.

\subsection{Ocsigen}

Ocsigen\cite{ocsigen-website} est le plus connu des frameworks web écrit en
OCaml.

Lancé en fin d'année 2004 par Vincent Balat, Ocsigen\footnote{Pour \og OCaml site
  generator \fg.} souhaite apporter des solutions, innovantes à ses débuts, à
plusieurs problématiques du développement d'applications web:

\begin{enumerate}
  \item utiliser un seul et unique langage (OCaml) pour développer la partie cliente et
    la partie serveur. En effet, la plupart des architectures utilisent deux
    langages différents. Cela implique qu'un développeur web doit apprendre différents
    langages. Du coté de l'entreprise, cela peut aussi impliquer de devoir
    embaucher deux développeurs: un pour le coté client, souvent appelé
    \og développeur front-end \fg, et un autre pour le coté serveur, souvent appelé
    \og développeur back-end \fg.
  \item décrire, dans un seul fichier, les morceaux de code qui seront
exécutés coté client et ceux qui seront exécutés coté serveur. Ceci est réalisé
à travers une extension de syntaxe PPX d'OCaml\footnote{Gabriel Radanne, aka
  Drup, travaille sur un compilateur Eliom, voir
  \cite{ocsigen-eliomlang-github}. Ce compilateur fournit plus de possibilités
  que PPX.}.
  \item injecter du code serveur dans la partie cliente.
    L'utilité de cette fonctionnalité est de laisser le serveur évaluer
des expressions pour décharger le client de certains calculs qui pourraient être
lourds.
  \item typer statiquement (au moment de la compilation, non à l'exécution), les
    programmes (coté serveur et coté client) ainsi que le code injecté.
  \item utiliser les bibliothèques du langage de base (OCaml), que cela
    soit coté client ou coté serveur.
\end{enumerate}

Bien que la plupart des points soient résolus indépendemment par plusieurs
technologies (Node.js\cite{nodejs-website}, Meteor\cite{meteor-website},
Scala\cite{scala-website} + Play\cite{play-website} +
ScalaJS\cite{scalajs-website}, entre autres), aucune technologie unique et
connue du grand public, répond à tous les besoins.

Le projet Ocsigen répond à chacun de ces besoins, et même plus. Le projet est
composé de plusieurs sous projets dont

\begin{itemize}
  \item Ocsigen Server\cite{ocsigen-server-github}, un serveur web écrit entièrement en OCaml.
  \item un compilateur de bytecode OCaml vers JavaScript, js\_of\_ocaml\cite{ocsigen-js-of-ocaml-github}, qui
    pemet d'écrire du code OCaml et de l'exécuter dans le navigateur web. En
plus du compilateur, js\_of\_ocaml fournit un binding OCaml assez complet (bien
que bas-niveau) au DOM et à la librairie standard JavaScript. Une particularité
de js\_of\_ocaml comparé à d'autres compilateurs/transpileurs OCaml vers
JavaScript est le support de tout le langage OCaml, sans syntaxe particulière.
  \item une bibliothèque implémentant le standard XML de manière typée en OCaml,
TyXML\cite{ocsigen-tyxml-github}. Cela permet d'écrire en particulier du HTML
typé et de respecter les normes du W3C.
  \item Eliom, qui permet d'écrire, avec un haut niveau
    d'abstraction et des concepts innovants, des applications web
client-serveur. Eliom dépend de Ocsigen Server, TyXML et js\_of\_ocaml et
fournit une abstraction plus haut niveau des API de ces différents projets.
  \item Ocsigen Toolkit, un ensemble de widgets web (différents boutons, des
    menus, des icones de chargements, etc) permettant d'utiliser les
    particularités d'exécution client-serveur d'Eliom.
  \item Ocsigen Start qui fournit un template Eliom avec un ensemble de
    bibliothèques pour gérer des utilisateurs, des groupes d'utilisateurs, des
    notifications push, etc. Ocsigen Start permet également de développer des
    applications mobiles, le tout en OCaml.
\end{itemize}

Ayant passé beaucoup de temps à apprendre et à travailler sur Eliom et Ocsigen
Start, l'apprentissage et la maitrise de ces deux bibliothèques ont eu une
grande importance lors de mon stage. 

\subsubsection{Eliom}

Le but principal d'Ocsigen est d'unifier le développement de la partie serveur
et la partie cliente en utilisant un langage unique et typé statiquement.

Eliom est le centre du projet Ocsigen (avec comme dépendances Ocsigen Server,
TyXML et js\_of\_ocaml) et constitue la solution aux besoins donnés ci-dessus.
Eliom fournit également une interface de haut
niveau pour des besoins usuels en programmation web comme un système de
notifications \\ (\verb|Eliom_notif|), de création d'HTML réactif
(\verb|Eliom_shared.React.S|) et des données réactives
(\verb|Eliom_shared.ReactiveData|), d'échange de données entre clients
(\verb|Eliom_bus|), de créations de services pour les routes de l'application et
des enregistrements de ces services (\verb|Eliom_service| et \\
\verb|Eliom_registration|), un système de gestion de cookies
\\ (\verb|Eliom_cookies|), création de formulaires typés
(\verb|Eliom_form|), un système de RPC (en utilisant la fonction
\verb|Eliom_client.server_function|) et bien plus.

De plus, depuis Eliom 6, il est possible d'écrire des applications mobiles.

En utilisant PPX, Eliom ajoute de la syntaxe au langage OCaml usuel.\footnote{Pour plus
d'informations sur le langage Eliom, voir \cite{gabriel-radanne-paper-eliom}.}
Eliom fournit également des outils pour simplifier la compilation:
\begin{itemize}
  \item \verb|eliomc| pour compiler du code serveur en bytecode.
  \item \verb|eliomopt| pour compiler du code serveur en natif.
  \item \verb|js_of_eliom| pour compiler en JavaScript le code client.
\end{itemize}

Pour différentier les fichiers contenant du code Eliom et ceux qui n'en \\
contiennent pas, de nouvelles extensions de fichiers sont utilisées: \verb|eliomi|
pour l'équivalent de \verb|mli| et \verb|eliom| pour \verb|ml|.

Pour définir une expression qui sera évaluée coté serveur (resp. coté client), la syntaxe
\verb|let%server v = 42| (resp. \verb|let%client v = 42|) est utilisée. Il
existe aussi la syntaxe \verb|let%shared v = 42| pour définir en même temps une
même expression coté client et coté serveur. L'équipe d'Ocsigen travaille
également sur un nouveau compilateur Eliom, voir
\cite{ocsigen-eliomlang-github}, qui fournit, entre autres, la même extension
pour les types (\verb|type%server t = int|) et le langage de modules
(\verb|module%server M = struct type t = int end|), non disponibles actuellement avec PPX.

Il est également possible de définir des valeurs coté serveur et d'y accéder
coté client avec la syntaxe \verb|let%client v = ~%server_value| où \\
\verb|server_value| est une valeur définie coté serveur.

Une dernière possibilité avec Eliom est, à l'inverse d'évaluer une expression
coté serveur et la récupérer coté client, il est possible, dans une expression
coté serveur, de calculer une expression coté client et d'en récupérer le
résultat coté serveur.
La syntaxe utilisée est \verb|let%server x = [%client 5 + 2]|.

Si plusieurs expressions doivent être évaluées coté serveur (resp. coté client),
il est possible d'utiliser la syntaxe \verb|[%%server expressions]| (resp. \\
\verb|[%%client expressions]|) dans les fichiers \verb|eliom| et la syntaxe \\
\verb|[%%server.start] expressions| (resp. \\ \verb|[%%client.start] expressions|)
dans les fichiers \verb|eliomi|.

\subsubsection{Ocsigen Start}

Anciennement eliom-base-app\footnote{Le nom fut changé pendant la période de
  mon stage.}, le projet Ocsigen Start\cite{ocsigen-start-github} est la base de
l'application Be Sport, distribué sur GitHub sous licence LGPL.

A travers ce projet, l'équipe d'Ocsigen souhaite fournir un ensemble de
fonctions permettant de créer très facilement un prototype d'une application web
et mobile. Cela implique, en partie, de gérer très facilement une base de
données d'utilisateurs, de groupes, d'emails, fournir des services usuels
(inscriptions d'utilisateurs, envoi d'email de confirmation, connexion,
modification de données personnelles), permettre de gérer plusieurs emails par
utilisateurs, le redimensionnement et cropping de photos et fournir des systèmes
de sessions et de cache.

En plus de la bibliothèque, Ocsigen Start fournit un template qui montre
comment utiliser les diverses fonctionnalités d'Eliom comme les
RPC\footnote{Remote Procedure Calls}, les notifications à travers
\verb|Eliom_notif| mais aussi les fonctions de la bibliothèque.

Le template fourni permet également d'avoir une application mobile (Android,
iOS, Windows Phone) de l'application web développée. Le framework web et mobile
Cordova est utilisé pour l'application Android et iOS tandis que le concept
d'Universal Web App est utilisé pour Windows Phone\footnote{Ceci en raison de
  la faible compatibilité des plugins Cordova sous Windows Phone.}.

Ocsigen Start réalise des choix pour les développeurs comme PostgreSQL pour la
base de données ou SASS (pré-processeur CSS) pour le style.
Dans les prochaines versions, Ocsigen Start devrait être compatible avec
différents gestionnaires de bases de données comme MySQL.