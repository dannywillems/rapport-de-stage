\section{En dehors des tâches}

Mon stage ne m'a pas apporté que des connaissances techniques.

\subsection*{Be Sport}

Mon stage ne s'arrêtait pas aux différentes tâches décrites. Chaque semaine,
nous avions une réunion avec l'ensemble de l'équipe afin de faire le point sur
l'avancée de chaque fonctionnalité. Ces réunions m'ont permis d'être plus
ordonné et d'arriver à me positionner dans l'avancement d'une tâche.

La tâche des notifications push et d'OAuth2.0 et OpenID Connect m'ont permis de
travailler sur des fonctionnalités en partant d'une idée jusqu'à la réalisation
concrète en passant par la recherche et l'étude de solutions existantes et les
spécifications.

Pour certaines de mes tâches, j'ai été mené à travailler avec d'autres membres
de l'équipe qui n'étaient pas nécessairement développeurs (designers,
rédacteurs, etc). J'ai pu réaliser que discuter avec des personnes n'ayant pas
de compétences en développement nécessite un travail sur la forme et sur la vulgarisation des concepts.

\subsection*{Erasmus +}

La ville de mon lieu de stage n'était pas un choix anodin. Je souhaitais découvrir
Paris pour son ambiance entrepreneuriale et pour être au centre des évolutions
technologiques.
Régulièrement, je me suis rendu à des conférences ou des meetups pour
développeurs et entrepreneurs.

Pendant tout mon séjour à Paris, j'habitais dans une HackerHouse \cite{hackerhouse-website}, un espace de
coliving entre développeurs et entrepreneurs. Je vivais avec 8 autres personnes
partageant les mêmes passions et les mêmes ambitions que moi. Nous réalisions des événements autour du
développement ou de l'entrepreneuriat au moins 2 week-ends par mois. Cela m'a
permis d'agrandir mon réseau professionnel ainsi que mes connaissances.
Vivre avec d'autres entrepreneurs m'a permis de prendre conscience de la réalité
de l'entrepreneuriat et de compléter le cours d'entrepreneuriat que j'ai suivi
lors de ma première année de master à l'université de Mons.

\subsection*{Autres}

A partir de septembre, j'ai également eu la chance de pouvoir assister les
lundis matins à un cours donné à l'Ecole Normale Supérieure (\og
$\lambda$-calcul et catégories \fg). Cela m'a permis d'aborder les langages
fonctionnels et en particulier OCaml avec un aspect plus théorique et de faire
le lien avec l'algèbre et les catégories.

De plus, j'ai pu me rendre à une conférence donné par Richard Stallman. Cette
rencontre m'a permis de me sensibiliser encore plus aux licences et au monde du
libre.